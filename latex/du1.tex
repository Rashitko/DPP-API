\documentclass{article}

\usepackage[czech]{babel}
\usepackage[utf8]{inputenc}
\usepackage[table]{xcolor}
\usepackage{chngpage}

\usepackage{hyperref}
\hypersetup{
    colorlinks,
    citecolor=black,
    filecolor=black,
    linkcolor=black,
    urlcolor=black
}

\title{\vspace{30mm}Doporučené postupy v programovaní - Návrh API\\Zhrňujúci text}
\author{Jakub Háva, Michal Raška}
\date{}


\addto\captionsenglish{% Replace "english" with the language you use
  \renewcommand{\contentsname}%
    {Obsah}%
}

\begin{document}

\maketitle
\thispagestyle{empty}

\clearpage
\tableofcontents

\clearpage
\section{Ako spustiť}

\section{Zdôvodnenie návrhu}
V tejto časti by sme radi uviedli dôvody, prečo sme sa rozhodli API navrhnúť tak ako sme ho navrhli a prečo sme zvolili tie postupy, ktoré sme zvolili. Nebudeme presne uvádzať všetky možné prípady ktoré mohli nastať v prezentovaných situáciach, na to slúži javadoc.
\subsection{Základná myšlienka}
Táto kapitola má za úlohu predviesť ideu celej knižnice. Tá je nasledovná:
\begin{itemize}
	\item užívateľ špecifikuje, aké možnosti môžu byť zadané na príkazovom riadku aplikácie, aké sú kladené obmedzenia na tieto možnosti, ktoré možnosti musia byť zadané a podobne
	\item následne je táto množina spracovaná a naplnená dátami, teda konkrétnymi hodnotami, ale aj dodatočnými informáciami ako napríklad, či sa splnili podmienky pre hodnotu možnosti
	\item nakoniec je užívateľovi prezentovaná možnosť získať všetky možnosti, ktoré kvôli niečomu neprešli (napríklad nebol splnený typ hodnoty, ktorý mal byť do možnosti dosadený), ale rovnako tak môže užívateľ aj získať všetky možnosti, ktoré nešpecifikoval, no na príkazovom riadku za ocitli (napríklad ich tam zadal užívateľ, lebo si neprečítal aké argumenty program akceptuje)
	\item rovako tak všetky možnosti majú po spracovaní nastavené hodnoty ktoré boli na príkazovom riadku nastavené, okrem tých kde hodnota nemohla byť nastavená, napríklad z dôvodu chybného typu hodnoty a zároveň sú nastavené všetky dodatočné informácie, ako napríklad, či sa splnili podmienky kladené na hodnotu
\end{itemize}
Ako je teda vidno, zvolili sme prístup, kedy si užívateľ musí aktívne zistiť požadované informácie. Druhým možným prístupom by bolo zvolenie princípu callbackov, no my sme sa rozhodli pre prvý spôsob, a to z dôvodu, že nám príde tento spôsob viac užívateľský príjemnejší, nakoľko užívateľ si určí kedy a ako chce s dátami a nie je obmedzovaný, ako by tomu bolo, aspoň v malej miere, v prípade callbackov.

\subsection{Triedna štruktúra}
Teraz by sme radi predstavili triednu štruktúru v rámci API. Tá nám bola trochu výtýkaná v jednej review, na základe čoho sme ju upravili, no myšlienku sme zachovali, lebo veríme, že je dobrá. Základom všetkého je trieda \texttt{Option}. Táto trieda predstavuje možnosť, ktorá môže byť nastavená rôznymi prepínačmi (krátkymi aj dlhými) a može a nemusí brať hodnotu a neskôr obsahuje aj informácie o tom, či je v poriadku (napríklad, či nenastala situácia, že je povinná, no na príkazovom riadku nebola špecifikovaná) a prípadne informáciu o tom, čo sa nepodarilo. Hodnotu predstavuje trieda \texttt{Argument}. Táto trieda obsahuje v sebe informácie o type, ako sa má naparsovať zo stringovej reprezentácie na príkazovom riadku (implementované pomocou triedy \texttt{ArgumentParser} a jej potomkov), obmedzeniach (implementované pomocou triedy \texttt{Constraint} a potomkov tejto triedy) a neskôr aj samotnú hodnotu, respektíve informácie prečo sa parsovanie nepodarilo.\\
Inštancie triedy \texttt{Option} sú potom uložené do \texttt{OptionsList}-u. Táto trieda zabezpečuje, aby sa žiadne dve možnosti neprekrývali v ani jednom prepínači. Existuje teda jednoznačné mapovanie z prepínaču (dlhého aj krátkeho), na najviac jednu inštanciu triedy \texttt{Option}.\\
\texttt{OptionsList} je následne odovzdaný triede \texttt{Parser}, ktorá vykoná samotné spracovanie dát na príkazovom riadku, nastaví potrebné údaje v triedach \texttt{Option} ktoré sú v \texttt{OptionsList}-e, a poskytne užívateľovi možnosti, ako sa dostať napríklad k regulárnym argumentom, ako vytiahnuť, všetky možnosti, ktoré zlyhali, opýtať sa či nejaké zlyhali...\\
Takéto rozvrhnutie tried nám umožnilo dúfame užívateľsky príjemným spôsobom definovať, čo je na príkazovom riadku očakávané a v akom formáte, vrátane užívateľsky definovaných formátov. Napríklad ak by bolo treba v nejakom prepínači spracovávať objekt JSON, tak to vôbec nie je problém. Vytvorí sa potomok triedy \texttt{ArgumentParser}, ktorý špecifikuje ako sa má string previesť na JSON a tento \texttt{ArgumentParser} sa nastaví triede \texttt{Argument}. Takto definovaný \texttt{Argument} sa nastaví do \texttt{Option}. Dostávame teda podľa nás pomerne rýchlo a jednoducho implementáciu prepínača, ktorý ako hodnotu očakáva objekt JSON.
\subsection{Ošetrovanie chýb}
Táto kapitola predstaví výnimky ktoré môžu pri bežnom používaní aplikácie vzniknúť. Tie vznikajú najčastejšie pri spracovávaní príkazového riadku, ktonkrétne v metódach ktoré parsujú stringovú reprezentáciu na požadovaný typ. V prípade, že v tomto procese bude vyhodená akákoľvek výnimka, je táto výnimka odchytená a miesto toho je vyhodená výnimka \texttt{ParsingException}
\subsection{Obmedzenia}

\end{document}