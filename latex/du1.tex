\documentclass{article}

\usepackage[czech]{babel}
\usepackage[utf8]{inputenc}
\usepackage[table]{xcolor}
\usepackage{chngpage}
\usepackage{fancyref}
\usepackage{textcomp}
\usepackage{longtable}

\usepackage{hyperref}
\hypersetup{
    colorlinks,
    citecolor=black,
    filecolor=black,
    linkcolor=black,
    urlcolor=black
}

\title{\vspace{30mm}Doporučené postupy v programovaní - Návrh API\\Zhrňujúci text}
\author{Jakub Háva, Michal Raška}
\date{}


\addto\captionsenglish{% Replace "english" with the language you use
  \renewcommand{\contentsname}%
    {Obsah}%
}

\begin{document}

\maketitle
\thispagestyle{empty}

\clearpage
\tableofcontents

\clearpage
\section{Ako spustiť}
Pre uľahčenie spustenia celého projektu, sme vytvorili Ant skript. V tomto skripte sú definované nasledujúce úlohy ("tasky"):
\begin{itemize}
    \item \texttt{clean} - vymaže vygenerované súbory
    \item \texttt{init} - vytvorí adresárovú štruktúru
    \item \texttt{compile} - zkompiluje knižnicu
    \item \texttt{jar} - vytvorí \texttt{.jar} súbor knižnice
    \item \texttt{compile-example} - skompiluje ukážkový príklad
    \item \texttt{run-example} - spustí ukážkový príklad
    \item \texttt{compile-tests} - skompiluje testy
    \item \texttt{run-tests} - spustí testy
    \item \texttt{build} - skompiluje \texttt{.jar} súbor knižnice, spustí testy a vygeneruje dokumentáciu
    \item \texttt{javadoc} - vygeneruje dokumentáciu
\end{itemize}

\clearpage
\section{Zdôvodnenie návrhu}
V tejto časti by sme radi uviedli dôvody, prečo sme sa rozhodli API navrhnúť tak ako sme ho navrhli a prečo sme zvolili tie postupy, ktoré sme zvolili. Nebudeme presne uvádzať všetky možné prípady ktoré mohli nastať v prezentovaných situáciach, ale len tie podľa nás najpodstatnejšie.

\subsection{Základná myšlienka}
Táto kapitola má za úlohu predviesť ideu celej knižnice. Tá je nasledovná:
\begin{itemize}
	\item užívateľ špecifikuje, aké možnosti môžu byť zadané na príkazovom riadku aplikácie, aké sú kladené obmedzenia na hodnoty týchto možnosti, ktoré možnosti musia byť zadané a podobne
	\item následne je táto množina spracovaná a naplnená dátami, teda konkrétnymi hodnotami, ale aj dodatočnými informáciami ako napríklad, či sa splnili podmienky pre hodnotu možnosti
	\item nakoniec je užívateľovi prezentovaná možnosť získať všetky možnosti, ktoré kvôli niečomu neprešli (napríklad nebol splnený typ hodnoty, ktorý mal byť do možnosti dosadený), ale rovnako tak môže užívateľ aj získať všetky možnosti, ktoré nešpecifikoval, no na príkazovom riadku za ocitli (napríklad ich tam zadal užívateľ, lebo si neprečítal aké argumenty program akceptuje)
	\item rovako tak všetky možnosti majú po spracovaní nastavené hodnoty ktoré boli na príkazovom riadku nastavené, okrem tých kde hodnota nemohla byť nastavená, napríklad z dôvodu chybného typu hodnoty a zároveň sú nastavené všetky dodatočné informácie, ako napríklad, či sa splnili podmienky kladené na hodnotu
\end{itemize}
Ako je teda vidno, zvolili sme prístup, kedy si užívateľ musí aktívne zistiť požadované informácie. Druhým možným prístupom by bolo zvolenie princípu callbackov, no my sme sa rozhodli pre prvý spôsob, a to z dôvodu, že nám príde tento spôsob viac užívateľský príjemnejší, nakoľko užívateľ si určí kedy a ako chce s dátami a nie je obmedzovaný, ako by tomu bolo, aspoň v malej miere, v prípade callbackov.

\subsection{Triedna štruktúra}
Teraz by sme radi predstavili triednu štruktúru v rámci API. Tá nám bola trochu výtýkaná v jednej review, na základe čoho sme ju upravili, no myšlienku sme zachovali, lebo veríme, že je dobrá. Základom všetkého je trieda \texttt{Option}. Táto trieda predstavuje možnosť, ktorá môže byť nastavená rôznymi prepínačmi (krátkymi aj dlhými) a može a nemusí brať hodnotu a neskôr obsahuje aj informácie o tom, či je v poriadku (napríklad, či nenastala situácia, že je povinná, no na príkazovom riadku nebola špecifikovaná) a prípadne informáciu o tom, čo sa nepodarilo. Hodnotu predstavuje trieda \texttt{Argument}. Táto trieda obsahuje v sebe informácie o type, ako sa má naparsovať zo stringovej reprezentácie na príkazovom riadku (implementované pomocou triedy \texttt{ArgumentParser} a jej potomkov), obmedzeniach (implementované pomocou triedy \texttt{Constraint} a potomkov tejto triedy) a neskôr aj samotnú hodnotu, respektíve informácie prečo sa parsovanie nepodarilo.\\
Inštancie triedy \texttt{Option} sú potom uložené do \texttt{OptionsList}-u. Táto trieda zabezpečuje, aby sa žiadne dve možnosti neprekrývali v ani jednom prepínači. Existuje teda jednoznačné mapovanie z prepínaču (dlhého aj krátkeho), na najviac jednu inštanciu triedy \texttt{Option}.\\
\texttt{OptionsList} je následne odovzdaný triede \texttt{Parser}, ktorá vykoná samotné spracovanie dát z príkazového riadku, nastaví potrebné údaje v triedach \texttt{Option} ktoré sú v \texttt{OptionsList}-e, a poskytne užívateľovi možnosti, ako sa dostať napríklad k regulárnym argumentom, ako vytiahnuť, všetky možnosti, ktoré zlyhali, opýtať sa či nejaké zlyhali...\\
Takéto rozvrhnutie tried nám umožnilo dúfame užívateľsky príjemným spôsobom definovať, čo je na príkazovom riadku očakávané a v akom formáte, vrátane užívateľsky definovaných formátov. Napríklad ak by bolo treba v nejakom prepínači spracovávať objekt JSON, tak to vôbec nie je problém. Vytvorí sa potomok triedy \texttt{ArgumentParser}, ktorý špecifikuje ako sa má string previesť na JSON a tento \texttt{ArgumentParser} sa nastaví triede \texttt{Argument}. Takto definovaný \texttt{Argument} sa nastaví do \texttt{Option}. Dostávame teda podľa nás pomerne rýchlo a jednoducho implementáciu prepínača, ktorý ako hodnotu očakáva objekt JSON.

\subsection{Ošetrovanie chýb}
Táto kapitola predstaví výnimky ktoré môžu pri bežnom používaní aplikácie vzniknúť. Tie vznikajú najčastejšie pri spracovávaní príkazového riadku, konkrétne v metódach ktoré parsujú stringovú reprezentáciu na požadovaný typ. V prípade, že v tomto procese bude vyhodená akákoľvek výnimka, je táto výnimka odchytená a miesto toho je vyhodená výnimka \texttt{ParsingException}. Táto výnimka je zachytená v knižnici a užívateľ ju nemusí chytať. V API sa prejaví tým, že \texttt{option.isFailed()} vrati \texttt{true} a \texttt{option.getParseResult()} vráti \\\texttt{Option.ParseResult.PARSING\_FAILED}.
Jediná chyba, okrem štandardných ako napríklad \texttt{NullPointerException} je chyba \texttt{IllegalArgumentException}, ktorá je vyhadzovaná v prípade, že užívateľ sa snaží vytvoriť inštanciu triedy \texttt{Option}, alebo pridať prepínač k už existujúcej, ktorý nie je validný. Tvar validneho prepínača je uvedený v kapitole \ref{Obmedzenia}. 
\subsection{Obmedzenia}
\label{Obmedzenia}
Validný prepínač musí spĺňať nasledovné kritéria:
\begin{itemize}
	\item na začiatku musí obsahovať neprázdnu sekvenciu písmen
	\item po tejto sekvencií môže nasledovať sekvencia písmen, čísel a znaku -, pričom znak - nesmie byť posledný
\end{itemize}
Platnými prepínačmi sú teda napríkla \texttt{short}, \texttt{s}, \texttt{option123}, \texttt{option-123}. Neplatnými sú napríklad prepínače \texttt{-s}, \texttt{123a}, \texttt{option-suffix-}. Treba mať na pamäti, že prepínače sa zadávajú bez úvedneho -, resp -{}- pri dlhých prepínačoch.\\
Ďalším obmedzením je hodnota argumentov. Okrem číselných hodnôt, sú všetky argumenty začínajúce znakom - považované za \texttt{Option} a teda buď sú prítomné v \texttt{OptionsList}-e, alebo sú vytvorené, pričom volanie \texttt{option.getParseResult()} na takto vytvorenej option vráti \texttt{Option.ParseResult.EXTRA}.\\
V prípade, že na na príkazovom riadku objaví reťazec začínajúci znakom -, napríklad \texttt{-{}-{}-{}not-existing}, pričom tento reťazec nie je číselný, udeje sa nasledovné:
\begin{itemize}
	\item najprv sa podľa počtu znakov - na začiatku rozhodne, či sa jedná o dlhý (aspoň -{}- na začiatku) alebo krátky prepínač (na začiatku je len znak -)
	\item ak je na začiatku viac ako dva znaky -, tak sa tieto znaky odstránia a zvyšok sa považuje za samotný prepínač, takže napríklad z \texttt{-{}-{}-{}not-existing} vznikne dlhý prepínač \texttt{-not-existing}
    \item ak je na začiatku jeden znak -, tak sa tento znak odstráni a zvyšok sa považuje za samotný prepínač, takže napríklad z \texttt{-not-existing} vznikne krátky prepínač \texttt{-not-existing}
	\item následne sa vytvorí \texttt{Option} pričom volanie \texttt{option.getParseResult()} na takto vytvorenej option vráti \texttt{Option.ParseResult.EXTRA}
\end{itemize}
Treba mať na pamäti, že takto vzniknuté \texttt{Option} nemajú validný prepínač, užívateľ však nemá možnosť vytvoriť Option s nevalidným prepínačom a takéto \texttt{Option} môžu vzniknúť len týmto spôsobom.

\subsection{Poznámky k správaniu sa knižnice v neštandardných podmienkach}
Uvedieme zopár podľa nás najdôležitejší príkladov neštandardných vstupov z príkazového riadku a to ako ich naša knižnica spracuje.
\begin{longtable}{|p{\dimexpr 0.30\linewidth-2\tabcolsep}|p{\dimexpr 0.70\linewidth-2\tabcolsep}|}
     \hline
     \textbf{Vstup} & \textbf{Spracovanie} \\
     \hline
     -short 1 123 & Hodnota argumentu možnosti \texttt{short} bude 1, a reťazec 123 bude vrátený v zozname, ktorý vráti volanie \texttt{parser.getUnmatchedArguments()} \\
     \hline
     -{}-{}-not-existing & Predpokladáme, že možnosť s prepínačom \texttt{-{}-{}-{}not-existing} nebola zadefinovaná, takže sa vytvorí nová, pričom volanie \texttt{option.getParseResult()} vráti \texttt{Option.ParseResult.EXTRA}. V kapitole \ref{Obmedzenia} je detailnejšie uvedené, ako sa takáto možnosť spracuje, nakoľko zápis jej prepínača nie je platný. \\
     \hline
     -short -{}-{}-not-existing & Hodnota možnosti \texttt{short} nie je definovaná, reťazec \texttt{-{}-{}-{}not-existing} je považovaný za prepínač, ale keďže nie je validný, vytvorí sa nová \texttt {Option}, ktorej sa tento reťazec nastaví ako dlhý prepínač, pričom volanie \texttt{option.getParseResult()} vráti \texttt{Option.ParseResult.EXTRA}. V kapitole \ref{Obmedzenia} je detailnejšie uvedené, ako sa takáto možnosť spracuje, nakoľko zápis jej prepínača nie je platný. \\
     \hline
     -short -{}-value & Hodnota argumentu možnosti \texttt{short} nie je definovaná, reťazec -{}-value je považovaný za dlhý prepínač, a bud existuje \texttt{Option} ktorá má takýto dlhý prepínač, alebo bude takáto \texttt{Option} vytvorená. Toto sa udeje aj v prípade, že možnosť \texttt{short} akceptuje stringový argument (v prípade, že je povinný, tak samozrejme volanie \texttt{option.getParseResult()} vráti \texttt{Option.ParseResult.ARGUMENT\_MISSING}) \\
     \hline
\caption{Tabuľka demonštrujúca spraávanie v neštandartných situáciach}
\end{longtable}
\end{document}